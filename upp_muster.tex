\documentclass[fontsize=12pt,paper=a4,DIV12,cleardoublepage=empty, 
liststotoc,idxtotoc,bibtotoc]{article}
\usepackage[ngerman]{babel}
\usepackage[utf8]{inputenc}
\usepackage[pdftex]{graphicx}
\usepackage{amsmath}
\usepackage{longtable}
\usepackage{stmaryrd}
\usepackage{colortbl}
\usepackage{eurosym}
\usepackage{amssymb}
\usepackage{amsthm}
\usepackage{nicefrac}
\usepackage{lscape}
\usepackage{pdfpages} 
%Seitenlayout
\renewcommand{\thesection}{\Roman{section}}
\renewcommand{\thesubsection}{\Roman{section}.\arabic{subsection}}
\usepackage[left=25mm, right=25mm, bottom=30mm]{geometry}
\usepackage[doublespacing]{setspace}
\usepackage[colorlinks=true, linkcolor=black, citecolor=black, urlcolor=blue]{hyperref} 
\usepackage[figure]{hypcap}
%Textstruktur
\usepackage{float}
\usepackage{wrapfig}
\usepackage{tabularx}
%Textaussehen
\usepackage{framed}
%Verzeichnisse
\usepackage{csquotes}
\usepackage[backend=biber, bibstyle=authoryear-ibid, uniquelist=false, maxbibnames=9, maxcitenames=3, citestyle=authoryear-icomp, isbn=true, url=true, block=space, pagetracker=page,   giveninits=false, dateabbrev=false, dashed=false ]{biblatex}
\addbibresource{MA_literatur.bib}
%\DefineBibliographyStrings{ngerman}{ 
%	andothers = {{et\,al\adddot}},             
%} 
\usepackage{etoolbox}
\apptocmd{\UrlBreaks}{\do\f\do\m}{}{}
\setcounter{biburllcpenalty}{9000}% Kleinbuchstaben
\setcounter{biburlucpenalty}{9000}% Großbuchsta
\expandafter\def\expandafter\quote\expandafter{\quote\small\singlespacing}

\setcounter{section}{0}
\setcounter{subsection}{0}

\newcommand*{\meincite}[1]{\citeauthor{#1} (\citeyear{#1})}

\newcommand{\KK}{\mathbb{•}{K}}
\newcommand{\CC}{\mathbb{C}}
\newcommand{\RR}{\mathbb{R}}
\newcommand{\QQ}{\mathbb{Q}}
\newcommand{\ZZ}{\mathbb{Z}}
\newcommand{\NN}{\mathbb{N}}
\newcommand{\PPO}{\mathcal{P}(\Omega)}

\theoremstyle{plain}
\newtheorem{satz}{Satz}[subsection]
\newtheorem{lem}[satz]{Lemma}
\newtheorem{theo}[satz]{Theorem}
\newtheorem{kor}[satz]{Korollar}
\newtheorem{defi}{Definition}
\theoremstyle{definition}
\newtheorem{bei}[satz]{Beispiel}
\newtheorem{bem}[satz]{Bemerkung}

\begin{document}
	\definecolor{gold}{rgb}{0.9,0.9,0}
	\begin{titlepage}
		\vspace*{-3cm}
		\noindent
		\hspace*{1cm}
			\begin{center}
				\centering
				{\LARGE Gesamtschule Scharnhorst}
			\end{center}
		\begin{center}
		\Large{Fundamentalsatz der Analysis}\\[0.5cm]
		\normalsize{geschrieben von}\\[0.25cm]	
		\large{Benno Schörmann}\\[0.5cm]
		\end{center}
	\begin{flushleft}
	\hyperref[subsec:thema1]{\textbf{\large Thema der Facharbeit:}}  \\
	\end{flushleft}
	Eine vollständige Definition und ein vollständiger Beweis des Fundamentalsatzes der Analysis
	\quad \\[1.5cm]
	\noindent 
	\renewcommand{\arraystretch}{1.4}
	\end{titlepage}
	\newpage
	\thispagestyle{empty}
	\tableofcontents
	\newpage
	%\listoffigures
	%\thispagestyle{empty}
	%\newpage	
	\section{Einleitung}
	(Die Einteilung steht übrigens noch absolut nicht fest, habe sie gestern abend im Halbschlaf angefertigt) \\\\
	In dieser Facharbeit werde ich über Den Fundamentalsatz der Analysis und die dazugehörigen Nebenpunkte schreiben.
	
	
	
	\section{Geschichtliche Zusammenfassung}
	Newton/Gauss Fight\\
	Jahreszeiten, Semi guter Beweis am Anfang?\\
	\footnote{Omar A. Hernandez Rodriguez (University of Puerto Rico) and Jorge M. Lopez Fernandez (University of Puerto Rico), "Teaching the Fundamental Theorem of Calculus: A Historical Reflection - Integration from Cavalieri to Darboux," Convergence (Januar 2012)}
	
	
	

	\section{Alle wichtigen Begriffe erklärt}
	Anschauliche Beispiele? (Scipy Einbindung?)
	
	
	\subsection{Was ist Differentialrechnung?}
	-eines der am einfachsten zu begreifenden Themen der Analysis ermöglicht dieser Teil der Analysis das finden von Extrema und das generelle Beschreiben von Funktionsverläufen.
	
	
	\subsubsection{Wie wird eine Funktion abgeleitet?}
		Text
	
	
	\subsection{Was ist Integralrechnung?}
		Text


	\subsubsection{Wie wird eine Funktion integriert?}
		Text
		
	
	\subsection{Was ist der Mittelwertsatz?}
		Text
		
	
	\subsection{Was ist eine stetige Funktion?}
		Text	
	
	
	
	
	\section{Hauptsatz der Differential- und Integralrechnung}
	
	Der Hauptsatz der Differential- und Integralrechnung ist in zwei Hauptsätze und einen Nebensatz aufgeteilt. Der erste Satz stellt den Zusammenhang zwischen Integral und Differential dar.

	\subsection{Erster Teil}
	\begin{satz}" "\\
		Gegeben ist die in $\RR$ definierte Funktion $f$ in einem geschlossenen Intervall $[a, b]$. Sei $F$ definiert für alle $x$ im Intervall $[a, b]$ durch \\
			\begin{equation*}
				F(x)=\int_{a}^{x}f(t) dt
			\end{equation*}
		Dann ist $F$ gleichmäßig stetig auf dem Intervall $[a, b]$ und differenzierbar auf dem offenen Intervall $(a, b)$, und 
			\begin{equation*}
				F'(x)=f(x)
			\end{equation*}
		Für alle $x$ in $(a, b)$, sodass $F$ eine Stammfunktion von $f$ ist.\\\\
	
	\end{satz}
	
	
	\begin{proof}" "\\
	
		Für ein gegebenes $f(t)$ sei die Funktion $F(x)$ definiert als
		\begin{equation*}
			F(x)=\int_{a}^{x}f(t)dt
		\end{equation*}
		Für jegliche 2 Zahlen $x_1$ und $\Delta x$ im Intervall $[a, b]$ ergibt sich
		\begin{equation*}
			F(x_1)=\int_{a}^{x_1}f(f)dt
		\end{equation*}
		und
		\begin{equation*}
			F(x_1+\Delta x)=\int_{a}^{x_1+\Delta x}f(t)dt
		\end{equation*}
		Wenn diese beiden Gleichungen nun subtrahiert werden, dann ergibt sich
		\begin{equation}
			F(x_1+\Delta x)-F(x_1)=\int_{a}^{x_1+\Delta x}f(t)dt-\int_{a}^{x_1}f(t)dt
		\end{equation}
		Die Summe beider Flächen ist
		\begin{equation}
			\int_{a}^{x_1}f(t)dt + \int_{x_1}^{x_1+\Delta x}f(t)dt = \int_{a}^{x_1+\Delta x}f(t)dt
		\end{equation}
		Die Umformung dieser Gleichungen gibt
		\begin{equation}
			\int_{a}^{x_1+\Delta x}f(t)dt-\int_{a}^{x_1}f(t)dt=\int_{x_1}^{x_1+\Delta x}f(t)dt
		\end{equation}
		Nun wird die Gleichung (1) eingesetzt.
		\begin{equation}
			F(x_1+\Delta x)-F(x_1)=\int_{x_1}^{x_1+\Delta x}f(t)dt
		\end{equation}
		Laut dem Mittelwertsatz gibt es eine Zahl c in $[x_1, x_1+\Delta x]$, sodass
		\begin{equation}
			\int_{x_1}^{x_1+\Delta x}f(t)dt=f(c)*\Delta x
		\end{equation}
		Nun wird die Gleichung (4) und (5) zusammengefügt und durch $\Delta x$ dividiert
		\begin{equation*}
			F(x_1+\Delta x)-F(x_1)=f(c)*\Delta x \;\;\;\;|\div \Delta x
		\end{equation*}
		\begin{equation*}
			\frac{F(x_1+\Delta x)-F(x_1)}{\Delta x}=f(c)
		\end{equation*}
		Auffallend ist, dass die linke Seite zu einem Differenzeinquotienten umgeformt wurde. Wird nun $\lim \limits_{\Delta x \to 0}$ angewandt, dann
		\begin{equation*}
			\lim \limits_{\Delta x \to 0} \frac{F(x_1+\Delta x)-F(x_1)}{\Delta x}=\lim \limits_{\Delta x \to 0}f(c)
		\end{equation*}
		Und somit
		\begin{equation}
			F'(x_1)=\lim \limits_{\Delta x \to 0}f(c)
		\end{equation}
		Nun fehlt nur noch $f(c)$.\\ Da $x_1 \leq c \leq x_1+\Delta x$ ist und $\Delta x$ gegen $0$ läuft wird sich $c\to$ $x_1$ nähern bis bei $\Delta x=0$ auch $c = x_1$ ist, beziehungsweise $x_1 \leq c \leq x_1 + 0$ oder $x_1 = c = x_1$ ist.\\Zusammenfassend ist also
		\begin{equation}
			\lim \limits_{\Delta x \to 0} f(c) = f(x_1)
		\end{equation}
		Und somit, wenn man (6) und (7) zusammenfügt
		\begin{equation*}
			F'(x_1)=f(x_1)
		\end{equation*}
		Eine andere Schreibweise wäre
		\begin{equation*}
			\frac{d}{dx}\int_{a}^{x}f(t)dt=f(x)
		\end{equation*}
		Damit ist der erste Satz der Differential- und Integralrechnung bewiesen.
	\end{proof}
	%\newpage
	
	\subsection{Korollar}
	\begin{satz}" "\\
		Gegeben ist die in $\RR$ definierte, auf dem Intervall $[a, b]$ stetige Funktion $f$  und $F$ eine Stammfunktion von $f$ im Intervall $[a, b]$, dann gilt
		\begin{equation*}
			\int_{a}^{b}f(t)dt=F(b)-F(a)
		\end{equation*}
		Das Korollar erfordert Stetigkeit auf dem ganzen Intervall.
	\end{satz}

	\begin{proof}" "\\
		Sei $F$ ein Stammfunktion von $f$ mit Stetigkeit auf dem Intervall $[a, b]$, dann sei
		\begin{equation*}
			G(x)=\int_{a}^{x}f(t)dt
		\end{equation*}
		Durch den ersten Beweis ist bewiesen, dass $G(x)$ eine Stammfunktion von f ist. Da $F'(x)-G'(x)=0$ ist der Mittelwertsatz impliziert, dass $F(x)-G(x)$ eine konstante Funktion ist, das heißt es gibt eine Zahl $c$, so dass $G(x)=F(x)+c$ für alle $x$ in $[a, b]$. Es gilt also
		\begin{equation*}
			F(a)+c=G(a)=\int_{a}^{a}f(t)dt=0
		\end{equation*}
		Das bedeutet, dass $c=-F(a)$. In anderen Worten, $G(x)=F(x)-F(a)$, und somit
		\begin{equation*}
			\int_{a}^{b}f(t)dt=G(b)=F(b)-F(a)
		\end{equation*}
	\end{proof}
		
		
	\subsection{Zweiter Teil}
	
	\begin{satz}" "\\
	Dieser Teil wird manchmal Zweiter Teil des Hauptsatzes der Differential- und Integralrechnung oder das Newton-Leibniz-Axiom genannt.\\
	Gegeben ist die in $\RR$ definierte Funktion $f$ auf dem geschlossenen Intervall $[a, b]$ und $F$ eine Stammfunktion von $f$ auf dem Intervall $(a, b)$
	\begin{equation*}
		F'(x)=f(x)
	\end{equation*}
	Wenn $f$ auf dem Intervall $[a, b]$ Riemann-integrierbar ist, dann 
	\begin{equation*}
		\int_{a}^{b}f(x)dx=F(b)-F(a)
	\end{equation*}
	Der zweite Teil ist aussagekräftiger als das Korollar, da er nicht voraussetzt, dass $f$ eine stetige Funktion ist.\\
	Wenn eine Stammfunktion $F$ von $f$ existiert, dann gibt es unendlich viele unterschiedliche Stammfunktionen die erlangt werden, indem eine Konstante, oftmals $c$ genannt, an $F$ addiert wird. Außerdem existiert laut dem ersten Teil eine Stammfunktion $F$, sobald $f$ kontinuierlich ist, was durch das nicht Vorhandensein einer Stammfunktion von Funktionen, wie $e^{-x^2}$ widerlegt ist.
	
	\end{satz}	
	
	\begin{proof} " "\\
	Dies ist ein Grenzbeweis durch die Riemann-Summe. Sei $f$ Riemann-integrierbar auf dem Intervall $[a, b]$ und lasse $f$ eine Stammfunktion $F$ auf dem Intervall $[a,b]$ zu. $F(b)-F(a)$. Seien $x_0 \cdots x_n$, sodass
	\begin{equation*}
		a = x_0 < x_1 < x_2 < \cdots < x_{n-2} < x_{n-1} < x_n = b
	\end{equation*}
	Daraus lässt sich erschließen, dass
	\begin{equation*}
		F(b)-F(a)=F(x_n)-F(x_0)
	\end{equation*}
	Nun wird jedes $F(x_k)$ zusammen mit seinem negativen gegenpart addiert, sodass die daraus ergebende menge folgendes ergibt
	\begin{multline*}
	\begin{aligned}
		F(b)-F(a)=& F(x_n)+[-F(x_{n-1})+F(x_{n-1})]+[-F(x_{n-2})+F(x_{n-2})]+\cdots +\\ &[-F(x_2)+F(x_2)]+[-F(x_1)+F(x_1)]-F(x_0)\\
		=& [F(x_n)-F(x_{n-1})]+[F(x_{n-1})-F(x_{n-2})]+\cdots+[F(x_2)-F(x_1)]+[F(x_1)-F(x_0)]
	\end{aligned}
	\end{multline*}
	Die obere Menge kann folgendermaßen zusammengefasst werden
	\begin{equation*}
		F(b)-F(a)=\sum_{k=1}^{n} [F(x_k)-F(x_{k-1})]
	\end{equation*}
	Als nächstes wird der Mittelwertsatz benötigt. Kurz zusammengefasst:\\
	Sei $F$ stetig auf dem geschlossenen Intervall $[a, b]$ und differenzierbar auf dem offenen Intervall $(a, b)$, dann existiert ein $c$ im Intervall $(a, b)$, sodass  [\cite{Lee2009a}]
	\begin{equation*}
		F'(c)=\frac{F(b)-F(a)}{b-a}
	\end{equation*}
	Daraus folgt
	\begin{equation*}
		F'(c)(b-a)=F(b)-F(a)
	\end{equation*}
	
	
	\end{proof}
	
	
	
	\section{Auswirkungen auf die Mathematik}
	
	
	
	\printbibliography[title=Literaturverzeichnis]


\end{document}