\documentclass[fontsize=13pt,paper=a4,DIV12,cleardoublepage=empty, 
liststotoc,idxtotoc,bibtotoc]{article}
\usepackage[ngerman]{babel}
\usepackage[utf8]{inputenc}
\usepackage[pdftex]{graphicx}
\usepackage{amsmath}
\usepackage{longtable}
\usepackage{stmaryrd}
\usepackage{colortbl}
\usepackage{eurosym}
\usepackage{amssymb}
\usepackage{amsthm}
\usepackage{nicefrac}
\usepackage{lscape}
\usepackage{pdfpages} 
%Seitenlayout
\renewcommand{\thesection}{\Roman{section}}
\renewcommand{\thesubsection}{\Roman{section}.\arabic{subsection}}
\usepackage[left=25mm, right=25mm, bottom=30mm]{geometry}
\usepackage[doublespacing]{setspace}
\usepackage[colorlinks=true, linkcolor=black, citecolor=black, urlcolor=blue]{hyperref} 
\usepackage[figure]{hypcap}
%Textstruktur
\usepackage{float}
\usepackage{wrapfig}
\usepackage{tabularx}
%Textaussehen
\usepackage{framed}
%Verzeichnisse
\usepackage{csquotes}
\usepackage[backend=biber, bibstyle=authoryear-ibid, uniquelist=false, maxbibnames=9, maxcitenames=3, citestyle=authoryear-icomp, isbn=true, url=true, block=space, pagetracker=page,   giveninits=false, dateabbrev=false, dashed=false ]{biblatex}
\addbibresource{MA_literatur.bib}
\DefineBibliographyStrings{ngerman}{ 
	andothers = {{et\,al\adddot}},             
} 
\usepackage{etoolbox}
\apptocmd{\UrlBreaks}{\do\f\do\m}{}{}
\setcounter{biburllcpenalty}{9000}% Kleinbuchstaben
\setcounter{biburlucpenalty}{9000}% Großbuchsta
\expandafter\def\expandafter\quote\expandafter{\quote\small\singlespacing}

\setcounter{section}{0}
\setcounter{subsection}{0}

\newcommand*{\meincite}[1]{\citeauthor{#1} (\citeyear{#1})}

\newcommand{\KK}{\mathbb{K}}
\newcommand{\CC}{\mathbb{C}}
\newcommand{\RR}{\mathbb{R}}
\newcommand{\QQ}{\mathbb{Q}}
\newcommand{\ZZ}{\mathbb{Z}}
\newcommand{\NN}{\mathbb{N}}
\newcommand{\PPO}{\mathcal{P}(\Omega)}

\theoremstyle{plain}
\newtheorem{satz}{Satz}[subsection]
\newtheorem{lem}[satz]{Lemma}
\newtheorem{theo}[satz]{Theorem}
\newtheorem{kor}[satz]{Korollar}
\newtheorem{defi}{Definition}
\theoremstyle{definition}
\newtheorem{bei}[satz]{Beispiel}
\newtheorem{bem}[satz]{Bemerkung}

\begin{document}
	\definecolor{gold}{rgb}{0.9,0.9,0}
	\begin{titlepage}
		\vspace*{-3cm}
		\noindent
		\hspace*{1cm}
		\begin{minipage}{0.05\textwidth}
			\begin{figure}[H]
				\centering 
				%\includegraphics[scale=0.2]{nrw.png}
			\end{figure}
		\end{minipage}
		\begin{minipage}{0.95\textwidth}
			\begin{center}
				{\LARGE Zentrum für schulpraktische Lehrerausbildung Siegen}\\
				Seminar für das Lehramt an Gymnasien und Gesamtschulen
			\end{center}
		\end{minipage}
		\\[0.5cm]
		\begin{center}
		\Large{Entwurf zur unterrichtspraktischen Prüfung im Fach Mathematik\footnote{gem. §32 (5) OVP vom 10.04.2011}}.\\[0.5cm]
		\normalsize{vorgelegt von}\\[0.25cm]	
		\large{Jan Heese}\\[0.5cm]
		\end{center}
	\begin{flushleft}
	\hyperref[subsec:thema1]{\textbf{\large Thema der Unterrichtsreihe:}}  \\
	\end{flushleft}
	Kinder in der Welt: Multiplikation und Division von Brüchen im interkulturellen Kontext. 
	\begin{flushleft}
	\hyperref[subsec:thema2]{\textbf{\large Thema der Unterrichtsstunde:}} 
	\end{flushleft}
	Stationenlernen: Produktives Üben zur Multiplikation und Division positiver Brüche mit Hilfe eines binnendifferenzierten Stationenlernens zur langfristigen Sicherung der Kompetenzen am Ende der Reihe. 
	\quad \\[1.5cm]
	\noindent 
	\renewcommand{\arraystretch}{1.4}
	\begin{tabular}{|l r|}
	\hline
	\hline 
	\end{tabular}
	\end{titlepage}
	\newpage
	\thispagestyle{empty}
	\tableofcontents
	\newpage
	%\listoffigures
	%\thispagestyle{empty}
	%\newpage
	\section{Längerfristige Unterrichtszusammenhänge}

\end{document}