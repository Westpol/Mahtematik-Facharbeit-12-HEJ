\documentclass[fontsize=12pt,paper=a4,DIV12,cleardoublepage=empty, 
liststotoc,idxtotoc,bibtotoc]{article}
\usepackage[ngerman]{babel}
\usepackage[utf8]{inputenc}
\usepackage[pdftex]{graphicx}
\usepackage{amsmath}
\usepackage{longtable}
\usepackage{stmaryrd}
\usepackage{colortbl}
\usepackage{eurosym}
\usepackage{amssymb}
\usepackage{amsthm}
\usepackage{nicefrac}
\usepackage{lscape}
\usepackage{pdfpages} 
%Seitenlayout
\renewcommand{\thesection}{\Roman{section}}
\renewcommand{\thesubsection}{\Roman{section}.\arabic{subsection}}
\usepackage[left=25mm, right=25mm, bottom=30mm]{geometry}
\usepackage[doublespacing]{setspace}
\usepackage[colorlinks=true, linkcolor=black, citecolor=black, urlcolor=blue]{hyperref} 
\usepackage[figure]{hypcap}
%Textstruktur
\usepackage{float}
\usepackage{wrapfig}
\usepackage{tabularx}
%Textaussehen
\usepackage{framed}
%Verzeichnisse
\usepackage{csquotes}
\usepackage[backend=biber, bibstyle=authoryear-ibid, uniquelist=false, maxbibnames=9, maxcitenames=3, citestyle=authoryear-icomp, isbn=true, url=true, block=space, pagetracker=page,   giveninits=false, dateabbrev=false, dashed=false ]{biblatex}
\addbibresource{MA_literatur.bib}
\DefineBibliographyStrings{ngerman}{ 
	andothers = {{et\,al\adddot}},             
} 
\usepackage{etoolbox}
\apptocmd{\UrlBreaks}{\do\f\do\m}{}{}
\setcounter{biburllcpenalty}{9000}% Kleinbuchstaben
\setcounter{biburlucpenalty}{9000}% Großbuchsta
\expandafter\def\expandafter\quote\expandafter{\quote\small\singlespacing}

\setcounter{section}{0}
\setcounter{subsection}{0}

\newcommand*{\meincite}[1]{\citeauthor{#1} (\citeyear{#1})}

\newcommand{\KK}{\mathbb{•}{K}}
\newcommand{\CC}{\mathbb{C}}
\newcommand{\RR}{\mathbb{R}}
\newcommand{\QQ}{\mathbb{Q}}
\newcommand{\ZZ}{\mathbb{Z}}
\newcommand{\NN}{\mathbb{N}}
\newcommand{\PPO}{\mathcal{P}(\Omega)}

\theoremstyle{plain}
\newtheorem{satz}{Satz}[subsection]
\newtheorem{lem}[satz]{Lemma}
\newtheorem{theo}[satz]{Theorem}
\newtheorem{kor}[satz]{Korollar}
\newtheorem{defi}{Definition}
\theoremstyle{definition}
\newtheorem{bei}[satz]{Beispiel}
\newtheorem{bem}[satz]{Bemerkung}

\begin{document}
	\definecolor{gold}{rgb}{0.9,0.9,0}
	\begin{titlepage}
		\vspace*{-3cm}
		\noindent
		\hspace*{1cm}
			\begin{center}
				\centering
				{\LARGE Gesamtschule Scharnhorst}
			\end{center}
		\begin{center}
		\Large{Fundamentalsatz der Analysis}\\[0.5cm]
		\normalsize{geschrieben von}\\[0.25cm]	
		\large{Benno Schörmann}\\[0.5cm]
		\end{center}
	\begin{flushleft}
	\hyperref[subsec:thema1]{\textbf{\large Thema der Facharbeit:}}  \\
	\end{flushleft}
	Eine vollständige Definition und ein vollständiger Beweis des Fundamentalsatzes der Analysis
	\quad \\[1.5cm]
	\noindent 
	\renewcommand{\arraystretch}{1.4}
	\end{titlepage}
	\newpage
	\thispagestyle{empty}
	\tableofcontents
	\newpage
	%\listoffigures
	%\thispagestyle{empty}
	%\newpage	
	\section{Einleitung}
	(Die Einteilung steht übrigens noch absolut nicht fest, habe sie gestern abend im Halbschlaf angefertigt) \\\\
	In dieser Facharbeit werde ich über Den Fundamentalsatz der Analysis und die dazugehörigen Nebenpunkte schreiben.
	
	
	
	\section{Geschichtliche Zusammenfassung}
	Newton/Gauss Fight\\
	Jahreszeiten, Semi guter Beweis am Anfang?\\
	\footnote{Omar A. Hernandez Rodriguez (University of Puerto Rico) and Jorge M. Lopez Fernandez (University of Puerto Rico), "Teaching the Fundamental Theorem of Calculus: A Historical Reflection - Integration from Cavalieri to Darboux," Convergence (January 2012)}
	
	
	

	\section{Alle wichtigen Begriffe erklärt}
	Anschauliche Beispiele? (Scipy Einbindung?)
	
	
	\subsection{Was ist Differentialrechnung?}
	-eines der am einfachsten zu begreifenden Themen der Analysis ermöglicht dieser Teil der Analysis das finden von Extrema und das generelle Beschreiben von Funktionsverläufen.
	
	
	\subsubsection{Wie wird eine Funktion abgeleitet?}
		Text
	
	
	\subsection{Was ist Integralrechnung?}
		Text


	\subsection{Wie wird eine Funktion integriert?}
		Text
		
	
	\subsection{Was ist der Mittelwertsatz?}
		Text
		
	
	\subsection{Was ist eine Stetige Funktion?}
		Text	
	
	
	
	
	\section{Hauptsatz der Differential- und Integralrechnung}
	
	Der Hauptsatz der Dieefrential- und Integralrechnung ist in zwei Hauptsätze und einen Nebensatz aufgeteilt. Der erste Satz stellt den Zusammenhang zwischen Integral und Differential dar.

	\subsection{Erster Satz}
	\begin{satz}
		Sei $f$ eine stetige, der Zahlenmenge $\RR$ angehörige Funktion in einem geschlossenen Intervall $[a, b]$. Sei $F$ definiert für alle $x$ im Intervall $[a, b]$ durch \\
			\begin{equation}
				F(x)=\int_{a}^{x}f(t) dt
			\end{equation}
		Dann ist $F$ gleichmäßig stetig auf dem Intervall $[a, b]$ und differenzierbar auf dem offenen Intervall $(a, b)$, und 
			\begin{equation}
				F'(x)=f(x)
			\end{equation}
		Für alle $x$ in $(a, b)$, sodass $F$ ein Gegendifferential von $f$ ist.\\\\

		Der Hauptsatz wird oftmals benutzt, um das Integral einer Funktion $f$ zu berechnen, dessen Gegendifferential $F$ bekannt ist. Wenn $f \in \RR$, kontinuierlich auf dem Intervall $[a, b]$ und $F$ ein Gegendifferential im Intervall $[a, b]$ ist, dann
		\begin{equation}
			\int_{a}^{b}f(t) dt=F(b)-F(a)
		\end{equation}
		Dieser Satz setzt Stetigkeit auf dem ganzen Intervall voraus.\\\\
	
	Dieser Satz wird oftmals $Zweiter Fundamentalsatz der Analysis$ oder Newton-Leibniz axiom genannt\\
	Sei $f \in \RR$ in einem geschlossenem Intervall $[a, b]$ und $F$ ein Gegendifferential von $f$ in $(a, b)$
	\begin{equation}
		F'(x)=f(x)
	\end{equation}
	Wenn $f$ Riemann-integrierbar in $[a, b]$ ist, dann gilt
	\begin{equation}
		\int_{a}^{b}f(t) dt = F(b)-F(a)
	\end{equation}
	
	\end{satz}
	
	
	\begin{proof}
	
		|\\Für ein gegebenes $f(t)$ sei die Funktion $F(x)$ definiert als
		\begin{equation}
			F(x)=\int_{a}^{x}f(t)dt
		\end{equation}
		Für jegliche 2 Zahlen $x_1$ und $\Delta x_1$ im Intervall $[a, b]$ ergibt sich
		\begin{equation}
			F(x_1)=\int_{a}^{x_1}f(f)dt
		\end{equation}
		und
		\begin{equation}
			F(x_1+\Delta x_1)=\int_{a}^{x_1+\Delta x}f(t)dt
		\end{equation}
		Wenn diese beiden Gleichungen nun subtrahiert werden, dann ergibt sich
		\begin{equation}
			F(x_1+\Delta x_1)-F(x_1)=\int_{a}^{x_1+\Delta x_1}f(t)dt-\int_{a}^{x_1}f(t)dt
		\end{equation}
		Die Summe beider Flächen ist
		\begin{equation}
			\int_{a}^{x_1}f(t)dt + \int_{x_1}^{x_1+\Delta x_1}f(t)dt = \int_{a}^{x_1+\Delta x_1}f(t)dt
		\end{equation}
		Die Umformung dieser Gleichungen gibt
		\begin{equation}
			\int_{a}^{x_1+\Delta x_1}f(t)dt-\int_{a}^{x_1}f(t)dt=\int_{x_1}^{x_1+\Delta x_1}f(t)dt
		\end{equation}
		Nun wird die Gleichung (n) eingesetzt.
		\begin{equation}
			F(x_1+\Delta x)-F(x_1)=\int_{x_1}^{x_1+\Delta x}f(t)dt
		\end{equation}
		Laut dem Mittelwertsatz gibt es eine Zahl c in $[x_1, x_1+\Delta x]$, sodass
		\begin{equation}
			\int_{x_1}^{x_1+\Delta x}f(t)dt=f(c)*\Delta x
		\end{equation}
		Nun wird die Gleichung (n) eingesetzt
		\begin{equation}
			F(x_1+\Delta x)-F(x_1)=f(c)*\Delta x
		\end{equation}
		Nun wird die Gleichung durch $\Delta x$ dividiert
		\begin{equation}
			\frac{F(x_1+\Delta x)-F(x_1)}{\Delta x}=f(c)
		\end{equation}
		Auffallend ist, dass auf die linke Seite zu einem Differenzeinquotienten umgeformt wurde. Wird nun $\lim \limits_{\Delta x \to 0}$ angewandt, dann
		\begin{equation}
			\lim \limits_{\Delta x \to 0} \frac{F(x_1+\Delta x)-F(x_1)}{\Delta x}=\lim \limits_{\Delta x \to 0}f(c)
		\end{equation}
		Und somit
		\begin{equation}
			F'(x_1)=\lim \limits_{\Delta x \to 0}f(c)
		\end{equation}
		Nun fehlt nur noch $f(c)$. Da $x_1 \leq c \leq x_1+\Delta x$ und $\Delta x$ gegen $0$ läuft wird sich $c\to$ $x_1$ nähern bis bei $\Delta x=0$ auch $c = x_1$ ist.\\Zusammenfassend ist also
		\begin{equation}
			\lim \limits_{\Delta x \to 0} f(c) = f(x_1)
		\end{equation}
		Und somit
		\begin{equation}
			F'(x_1)=f(x_1)
		\end{equation}
		Damit ist der erste Satz der Differential- und Integralrechnung bewiesen.
	\end{proof}
	
	\subsection{Korollar}
		
	
	\section{Auswirkungen auf die Mathematik}
	Brüche kann man auf diese weise kürzen:
		\begin{equation}
			\begin{aligned}
				F(x)&=\frac{{x}^{4}}{30}+\frac{2}{5}{x}^{3}\\
				f(x)&=\frac{{x}^{3}}{10}+5{x}^{2}\\
				f'(x)&=\frac{3}{10}{x}^{2}+10x\\
				f''(x)&=\frac{6}{10}x+10\\
				f'''(x)&=\frac{6}{10}\\
				f''''(x)&=0\\
			\end{aligned}
		\end{equation}


\end{document}