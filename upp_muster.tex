\documentclass[fontsize=12pt,paper=a4,DIV12,cleardoublepage=empty, 
liststotoc,idxtotoc,bibtotoc]{article}
\usepackage[ngerman]{babel}
\usepackage[utf8]{inputenc}
\usepackage[pdftex]{graphicx}
\usepackage{amsmath}
\usepackage{longtable}
\usepackage{stmaryrd}
\usepackage{colortbl}
\usepackage{eurosym}
\usepackage{amssymb}
\usepackage{amsthm}
\usepackage{nicefrac}
\usepackage{lscape}
\usepackage{pdfpages} 
%Seitenlayout
\renewcommand{\thesection}{\Roman{section}}
\renewcommand{\thesubsection}{\Roman{section}.\arabic{subsection}}
\usepackage[left=25mm, right=25mm, bottom=30mm]{geometry}
\usepackage[doublespacing]{setspace}
\usepackage[colorlinks=true, linkcolor=black, citecolor=black, urlcolor=blue]{hyperref} 
\usepackage[figure]{hypcap}
%Textstruktur
\usepackage{float}
\usepackage{wrapfig}
\usepackage{tabularx}
%Textaussehen
\usepackage{framed}
%Verzeichnisse
\usepackage{csquotes}
\usepackage[backend=biber, bibstyle=authoryear-ibid, uniquelist=false, maxbibnames=9, maxcitenames=3, citestyle=authoryear-icomp, isbn=true, url=true, block=space, pagetracker=page,   giveninits=false, dateabbrev=false, dashed=false ]{biblatex}
\addbibresource{MA_literatur.bib}
\DefineBibliographyStrings{ngerman}{ 
	andothers = {{et\,al\adddot}},             
} 
\usepackage{etoolbox}
\apptocmd{\UrlBreaks}{\do\f\do\m}{}{}
\setcounter{biburllcpenalty}{9000}% Kleinbuchstaben
\setcounter{biburlucpenalty}{9000}% Großbuchsta
\expandafter\def\expandafter\quote\expandafter{\quote\small\singlespacing}

\setcounter{section}{0}
\setcounter{subsection}{0}

\newcommand*{\meincite}[1]{\citeauthor{#1} (\citeyear{#1})}

\newcommand{\KK}{\mathbb{•}{K}}
\newcommand{\CC}{\mathbb{C}}
\newcommand{\RR}{\mathbb{R}}
\newcommand{\QQ}{\mathbb{Q}}
\newcommand{\ZZ}{\mathbb{Z}}
\newcommand{\NN}{\mathbb{N}}
\newcommand{\PPO}{\mathcal{P}(\Omega)}

\theoremstyle{plain}
\newtheorem{satz}{Satz}[subsection]
\newtheorem{lem}[satz]{Lemma}
\newtheorem{theo}[satz]{Theorem}
\newtheorem{kor}[satz]{Korollar}
\newtheorem{defi}{Definition}
\theoremstyle{definition}
\newtheorem{bei}[satz]{Beispiel}
\newtheorem{bem}[satz]{Bemerkung}

\begin{document}
	\definecolor{gold}{rgb}{0.9,0.9,0}
	\begin{titlepage}
		\vspace*{-3cm}
		\noindent
		\hspace*{1cm}
		\begin{minipage}[c]{.85\textwidth}
			\begin{center}
				\centering
				{\LARGE Gesamtschule Scharnhorst}
			\end{center}
		\end{minipage}
		\\[0.5cm]
		\begin{center}
		\Large{Fundamentalsatz der Analysis}\\[0.5cm]
		\normalsize{geschrieben von}\\[0.25cm]	
		\large{Benno Schörmann}\\[0.5cm]
		\end{center}
	\begin{flushleft}
	\hyperref[subsec:thema1]{\textbf{\large Thema der Facharbeit:}}  \\
	\end{flushleft}
	Eine vollständige Definition und ein vollständiger Beweis des Fundamentalsatzes der Analysis
	\quad \\[1.5cm]
	\noindent 
	\renewcommand{\arraystretch}{1.4}
	\end{titlepage}
	\newpage
	\thispagestyle{empty}
	\tableofcontents
	\newpage
	%\listoffigures
	%\thispagestyle{empty}
	%\newpage	
	\section{Einleitung}
	(Die Einteilung steht übrigens noch absolut nicht fest, habe sie gestern abend im Halbschlaf angefertigt) \\\\
	In dieser Facharbeit werde ich über Den Fundamentalsatz der Analysis und die dazugehörigen Nebenpunkte schreiben.
	
	
	\subsection{Bewegtgründe?}
	Fand das Thema interressant etc.
	
	
	\section{Geschichtliche Zusammenfassung}
	Newton/Gauss Fight\\
	Jahreszeiten 
	
	

	\section{Alle wichtigen Begriffe erklärt}
	Anschauliche Beispiele? (Scipy Einbindung?)
	
	
	\subsection{Was ist Differentialrechnung?}
	-eines der am einfachsten zu begreifenden Themen der Analysis ermöglicht dieser Teil der Analysis das finden von Extrema und das generelle Beschreiben von Funktionsverläufen.
	
	
	\subsubsection{Wie wird eine Funktion abgeleitet?}
		Text
	
	
	\subsection{Was ist Integralrechnung?}
		Text


	\subsection{Wie wird eine Funktion integriert?}
		Text
		
	
	\subsection{Was ist der Mittelwertsatz?}
		Text
	
	
	
	\section{Hauptsatz der Differential- und Integralrechnung}
	
	\begin{theo}
		Text
	\end{theo}
	
	\begin{satz}
		Text
	\end{satz}
	
	\begin{defi}
		Text
	\end{defi}
	
	Benno Barnes
	Nicki Schörmann

\end{document}