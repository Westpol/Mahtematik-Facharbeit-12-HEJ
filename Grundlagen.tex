\documentclass[ngerman]{article}

\usepackage[T1]{fontenc}
\usepackage[utf8]{inputenc}
\usepackage{babel}
\pagestyle{empty}

\begin{document}
	\section{Einführung}
	Hier ist ein etwas längerer deutscher Beispieltext: Durch
	Angabe der Option \texttt{ngerman} werden u.a. mit \texttt{babel}
	deutsche Trennmuster verwendet. Das Paket \texttt{inputech} wird
	mit der Option \texttt{latin1} geladen und sorgt für die passende
	Eingabecodierung \scriptsize{(Umlaute)}\normalsize.
	
	Erstmal noch ein neuer Absatz.
	\subsection{Meine Bewegtgründe}
	Und hier beginnt ein neuer Unterpunkt.
	
	\section{Annahme 1}
	lul, nice
	\subsection{Satz 1}
	\subsection{Beweis 1}
	
\end{document}
